\section{Erklären leicht gemacht}

\subsection{Top-Down-Methode}
Diese Methode eignet sich sowohl für den gesamten Termin als auch für eine einzelne Einheit.

\begin{quote}
  \emph{Tell 'em what you gonna say,\\
  say it,\\
  tell 'em what you've just said.}
\end{quote} 

Mit anderen Worten:
\begin{enumerate}
  \item Fahrplan vorstellen
  \item machen
  \item zusammenfassen
\end{enumerate}

\subsection{Sandwich-Methode}
Diese Methode eignet sich, um komplizierte Sachverhalte zu erklären.

\begin{enumerate}
  \item einfaches Beispiel, dass den Sachverhalt veranschaulicht (möglichst ohne neue Begriffe)
  \item allgemeine Formulierung, Definition, Fachbegriffe
  \item komplexeres Beispiel mit Fachbegriffen
\end{enumerate}

Wichtig ist dabei, Fachbegriffe \emph{vor} der ersten Benutzung zu definieren.