\chapter{Domain models}

\section{Naming}

\subsection{Singular/Plural}

Use the singular with CamelCase, not plural.

Bad:

\begin{phpcode}
class OliverKlee\Books\Domain\Model\Users {…}
\end{phpcode}

Good:

\begin{phpcode}
class OliverKlee\Books\Domain\Model\User {…}
\end{phpcode}


\subsection{DDD contexts as namespaces}

If you have a lot of classes, practice good domain-driven design (DDD), group you model by contexts, and use the contexts as sub-namespaces.

Good:
\begin{itemize}
  \item \texttt{Domain\textbackslash Model\textbackslash Identity\textbackslash Organization}
  \item \texttt{Domain\textbackslash Model\textbackslash Identity\textbackslash User}
  \item \texttt{Domain\textbackslash Model\textbackslash Place\textbackslash PostalAddress}
  \item \texttt{Domain\textbackslash Model\textbackslash Workshops\textbackslash Tag}
  \item \texttt{Domain\textbackslash Model\textbackslash Workshops\textbackslash Workshop}
\end{itemize}


\subsection{Method names}

Only getters should be named get*. Other methods that are not regular getters should be named differently, e.\,g., calculate*, retrieve*, determine* etc.

Bad:

\begin{phpcode}
public function getMostRecentItem($index) {…}
\end{phpcode}

Good:

\begin{phpcode}
public function retrieveMostRecentItem($index) {…}
\end{phpcode}


\section{Queries}

Don't use repositories, SQL or queries in your models. Instead, use relations (usually, with lazy loading). Or move the corresponding method into a repository

Bad:

\begin{phpcode}
class Brochure extends AbstractEntity
{
    public function getSubCategories()
    {
        $programCategoryRepository = $this->objectManager->get(
          ProgramCategoryRepository::class
        );

        return $programCategoryRepository->findAllSubCategoriesByMainCategory(
          $this->getFirstCategory()
        );
    }
}
\end{phpcode}

Instead of creating getters that iterate over data and do repository calls for these, use find methods in repositories.


\section{Associations}

\subsection{Lazy associations}

Always use \texttt{@lazy} for your associations. The only exception is if you \emph{always} use the association if you use the model.

If you use lazy n:1 or 1:1 associations and you want to pass the return value of the getter into a type-hinted method, you'll need to resolve the lazy loading in the getter:

\begin{phpcode}
/**
  * @var \OliverKlee\Books\Domain\Model\Group
  * @lazy
  */
protected $group = null;

public function getGroup()
{
    if ($this->group instanceof LazyLoadingProxy) {
        $this->group = $this->group->_loadRealInstance();
    }

    return $this->group;
}
\end{phpcode}


\subsection{Count methods on collections}
\label{relation-count}

Calling \texttt{count} on a collection results in a query each time it is called. This is even the case if the association has been iterated over before.

If your application is performance-critical, consider fetching the count directly from the relation counter cache. You'll need to update the cache each time the relation is set or an item is added or removed, though.

Example:

\begin{phpcode}
trait CachedRelationCount
{
    /** @var int[] */
    protected $cachedRelationCountsCount = [];

    /**
     * @param string $propertyName relation name (plural, lower camelCase)
     * @return int
     */
    protected function getCachedRelationCount($propertyName)
    {
        if (array_key_exists($propertyName, $this->cachedRelationCountsCount)) {
            return $this->cachedRelationCountsCount[$propertyName];
        }

        $this->cachedRelationCountsCount[$propertyName]
          = $this->getUncachedRelationCount($propertyName);

        return $this->cachedRelationCountsCount[$propertyName];
    }

    /**
     * @param string $propertyName relation name (plural, lower camelCase)
     * @return void
     */
    protected function flushRelationCountCache($propertyName)
    {
        unset($this->cachedRelationCountsCount[$propertyName]);
    }

    /**
     * Retrieves the relation count for the given property.
     *
     * This methods tries to avoid database accesses by using the relation
     * counter cache if the relation is still a LazyObjectStorage. Otherwise,
     * the normal COUNT query will be performed as a fallback.
     *
     * This method does not cache its results.
     *
     * @param string $propertyName relation name (plural, lower camelCase)
     * @return int
     */
    protected function getUncachedRelationCount($propertyName)
    {
        if ($this->$propertyName instanceof LazyObjectStorage) {
            $reflectionClass = new \ReflectionClass(LazyObjectStorage::class);
            $reflectionProperty = $reflectionClass->getProperty('fieldValue');
            $reflectionProperty->setAccessible(true);
            $count = (int)$reflectionProperty->getValue($this->$propertyName);
        } else {
            $count = $this->$propertyName->count();
        }

        return $count;
    }
}
\end{phpcode}

The application in a model then looks like this:

\begin{phpcode}
public function setThreads(ObjectStorage $threads)
{
    $this->flushThreadsCountCache();
    $this->threads = $threads;
}

public function getNumberOfThreads()
{
  return $this->getCachedRelationCount('threads');
}

public function addThread(Thread $thread)
{
    $this->flushThreadsCountCache();
    $this->threads->attach($thread);
}

public function removeThread(Thread $threadToRemove)
{
    $this->flushThreadsCountCache();
    $this->threads->detach($threadToRemove);
}
\end{phpcode}

In the fluid template, you then use \texttt{item.numberOfThreads} instead of \texttt{item.threads.count}:

\begin{htmlcode}
<p>
  Number of threads: {item.numberOfThreads}
</p>
\end{htmlcode}


\subsection{Association type hinting}
Always use the fully-qualified class name (FQDN) for the type annotation for attributes. (Extbase is not namespace-aware in this regard.)

You'll need to stop PhpStorm from over-eagerly using the short class name for you.

Bad:

\begin{phpcode}
use \OliverKlee\Books\Domain\Model\Group;

class Forum {
  /**
    * @var Group
    * @lazy
    */
  protected $group = null;
}
\end{phpcode}

Good:

\begin{phpcode}
class Forum {
  /**
    * @var \OliverKlee\Books\Domain\Model\Group
    * @lazy
    */
  protected $group = null;
}
\end{phpcode}


\section{Traits}

If you have the same attribute(s) and the corresponding getters/setters in multiple models, consider either creating a common subclass or an interface plus Traits. Which version (if any) is right depends on what these attributes mean semantically.

Examples for traits:

\begin{itemize}
  \item \texttt{Authored} for models that have an association to an author
  \item \texttt{CreationDatable} for models that have a getter for the creation date
  \item \texttt{CachedRelationCount} for providing the code for accessing a cached relation count (as mentioned on page~\pageref{relation-count})
\end{itemize}


