\chapter{Repositories}


\section{Naming}

\subsection{Singular/Plural}

Use the singular with CamelCase, not plural.

Bad:

\begin{phpcode}
class OliverKlee\Books\Domain\Repository\UsersRepository {…}
\end{phpcode}

Good:

\begin{phpcode}
class OliverKlee\Books\Domain\Repository\UserRepository {…}
\end{phpcode}


\section{Performance}


\subsection{Storage page}
If you don’t need the storage page, setRespectStoragePage(false) for better performance:

\begin{phpcode}
public function initializeObject()
{
  /** @var QuerySettingsInterface $querySettings */
  $querySettings = $this->objectManager->get(QuerySettingsInterface::class);
  $querySettings->setRespectStoragePage(false);
  $this->setDefaultQuerySettings($querySettings);
}
\end{phpcode}


\subsection{Sorting}

If you don’t need sorting, don’t set any sorting/ordering (for better performance). There is no code to add for this.


\subsection{Sub-queries}

If possible, move the \texttt{find*} method to another repository altogether. In this example, the method now is in the \texttt{GroupMembershipRepository} instead of the \texttt{GroupRepository}:

\begin{phpcode}
class GroupMembershipRepository extends Repository
{
  public function findByGroup(Group $group)
  {
    $query = $this->createQuery();
    $query->matching($query->equals('group', $group))

    return $query->execute();
  }
}
\end{phpcode}

You can also use queries that create JOINs instead of querying other repositories in a loop:

\begin{phpcode}
class GroupMembershipRepository extends Repository
{
  public function findByGroupName(string $groupName)
  {
    $query = $this->createQuery();
    $query->matching($query->equals('group.name', $name))

    return $query->execute();
  }
}
\end{phpcode}


\section{Safe LIKE queries}

Make sure to always correctly escape in LIKE queries to avoid SQL injections.

Bad:

\begin{phpcode}
public function findBySearchTerm($term)
{
  $query = $this->createQuery();
  $query->like($searchField, '%' . $searchTerm . '%', false);

  return $query->execute();
}
\end{phpcode}

Good:

\begin{phpcode}
public function findBySearchTerm($term)
{
  $safeSearchTerm = $this->getDatabaseAdapter()
    ->escapeStrForLike($searchTerm, $this->tableName);

  $query = $this->createQuery();
  $query->like($searchField, '%' . $safeSearchTerm . '%', false);

  return $query->execute();
}
\end{phpcode}
