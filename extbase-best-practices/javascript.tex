\chapter{JavaScript}
\label{javascript}

\section{Including the JavaScript files}

Include your extension JavaScript in the page bottom, not in the HEAD. This keeps the JavaScript from blocking the HTML parsing.

Bad:

\begin{textcode}
page {
  includeJS {
    books = EXT:books/Resources/Public/JavaScript/FrontEnd/FrontEnd.js
  }
}
\end{textcode}

Good:

\begin{textcode}
page {
  includeJSFooter {
    books = EXT:books/Resources/Public/JavaScript/FrontEnd/FrontEnd.js
  }
}
\end{textcode}

Note: This approach makes it necessary that there is no inline JavaScript in your page that requires the JavaScript file.


\section{Event handlers}
Use your JavaScript file to attach any necessary event handlers to your DOM. Don't have any \texttt{onclick} handlers etc.~in your HTML.

Bad:

\begin{htmlcode}
<button type="button" onclick="TYPO3.books.submitSearchForm();">
  Search
</button>
\end{htmlcode}

Good:

\begin{jscode}
jQuery(document).ready(function () {
    if (jQuery('.tx-books-pi1').length === 0) {
        return;
    }

    TYPO3.books.initializeSearchWidget();
    TYPO3.books.initializeRegistrationForm();
    TYPO3.books.convertActionLinks();
});
\end{jscode}


\section{Global names}

Don’t use global variables or methods. Always namespace them or put them in modules to avoid polluting the global namespace (and to avoid naming collisions).
