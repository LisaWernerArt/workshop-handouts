%----------------------------------------------------------------------------------
% Pakete und Parameter
%----------------------------------------------------------------------------------

% Input-Encoding für UTF-8
\usepackage[utf8]{inputenc}

% Mehr von der Seitenbreite nutzen
\usepackage{a4wide}

% Grafikpaket
\usepackage{color}
\usepackage[pdftex]{graphicx}

% Absätze werden nicht eingezogen, sondern vertikal abgesetzt
\setlength{\parindent}{0mm}
\addtolength{\parskip}{0.4em}

% Palatino und Helvetica statt Computer Modern als Standard-Fonts
\usepackage{palatino}

% Bibliographieeinstellungen
\usepackage{natbib}
\bibliographystyle{alpha}

% lesbare Verweise
\usepackage[pdftex,plainpages=false,pdfpagelabels]{hyperref}

% Fonts in PDF hübsch machen
\usepackage{ae}

% nette URLs
\usepackage{url}

% für Boxen etc.
\usepackage{framed}
\definecolor{shadecolor}{rgb}{0.8,0.8,0.8}

% Anführungszeichen sprachenabhängig machen
\usepackage[babel]{csquotes}


%----------------------------------------------------------------------------------
% Seitenlayout
%----------------------------------------------------------------------------------

% Seiten-Kopfzeilen und -Fußzeilen
\usepackage{fancyhdr}
\pagestyle{fancy}
\fancyhf{}
\fancyhead[RE]{\slshape \nouppercase{\leftmark}}    % links:  "Seite      Kapitel"
\fancyhead[LO]{\slshape \nouppercase{\rightmark}}   % rechts: "Kapitel    Seite"
\fancyhead[RO,LE]{\bfseries \thepage}
\renewcommand{\headrulewidth}{1pt} % Kopfzeilen unterstreichen
\renewcommand{\footrulewidth}{0pt}

\fancypagestyle{plain}{ % Keine Kapitel und Abschnitt auf Startseite start pages
\fancyhf{}
\fancyhead[RO,LE]{\bfseries \thepage}
\renewcommand{\headrulewidth}{1pt}
\renewcommand{\footrulewidth}{0pt}
}

% Kopfzeile auf linker Seite: "1  Einführung"
\renewcommand{\chaptermark}[1]{%
\markboth{\thechapter\ \ \ \ #1}{}}

% Kopfzeile auf rechter Seite: "1.1  Basics"
\renewcommand{\sectionmark}[1]{%
\markright{\thesection\ \ \ \ #1}{}}

% Seitenlayout
\topmargin0mm
\addtolength{\headheight}{2pt} % verhindert zu volle vboxes durch fancyhdr
\footskip10mm % Abstand von unserem Rand zu Datum

\newlength{\fullwidth} % Seites des Textes plus Randnotizen
\setlength{\fullwidth}{\textwidth}
\addtolength{\fullwidth}{\marginparsep}
\addtolength{\fullwidth}{\marginparwidth}

% Maximal drei Ebenen nummerieren
\setcounter{secnumdepth}{3}
% Maximale Gliederungstiefe, die noch ins Inhaltsverzeichnis aufgenommen wird
\setcounter{tocdepth}{1}

% zweispaltiges Layout möglich machen
\usepackage{multicol}

% Descriptions ohne Einzug
\renewenvironment{description}[1][0pt]
{\list{}{
    \labelwidth=0pt \leftmargin=#1
     \let\makelabel\descriptionlabel
  }
}
{\endlist}

\raggedbottom

\newcommand{\fett}[1]{\textsf{\textbf{#1}}}