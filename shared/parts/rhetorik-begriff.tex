\section{Was ist Rhetorik?}
\paragraph*{Redeschulung ist Denkschulung:} Gute Reden verlangen strukturierte Gedanken und setzen die Fähigkeit zum sauberen Argumentieren voraus.
\paragraph*{Überzeugendes Reden setzt Überzeugung voraus:} Glaubwürdigkeit ist der entscheidende Maßstab für jegliche Rhetorik. Sie bedeutet, dass Form und Inhalt nicht zu trennen sind.
\paragraph*{Reden und Redlichkeit gehören zusammen:} Wer redet, übernimmt Verantwortung für die Wirkung seiner Rhetorik~-- und hat deswegen mit den rhetorischen Mitteln verantwortlich umzugehen!

\section{Aufgaben der RednerIn}
\subsubsection{Informieren}
Sachverhalte, Probleme, Situationen, Vorgänge, Abläufe, \ldots

\subsubsection{Überzeugen}
Vorgehensweisen, Entscheidungen, Lösungen, \ldots

\subsubsection{Unterhalten}
Lachen, Schmunzeln, Nachdenken, Erinnern, \ldots
