\section{Feedback: Tipps und Tricks}
\index{Feedback}
\subsection{Was ist Feedback?}
Feedback ist für euch eine Gelegenheit, in kurzer Zeit viel über euch selbst zu lernen. Feedback ist ein Anstoß, damit ihr danach an euch arbeiten könnt (wenn ihr wollt).

Feedback heißt, dass euch jemandem seinen/ihren persönlichen, subjektiven Eindruck in Bezug auf konkrete Punkte mitteilt. Da es sich um einen persönlichen Eindruck im Kopf eines einzelnen Menschen handelt, sagt Feedback nichts darüber aus, wie ihr tatsächlich wart. Es bleibt allein euch selbst überlassen, das Feedback, das ihr bekommt, für euch selbst zu einem großen Gesamtbild zusammenzusetzen.

Es kann übrigens durchaus vorkommen, dass ihr zur selben Sache von verschiedenen Personen völlig unterschiedliches (oder gar gegensätzliches) Feedback bekommt.

Es geht beim Feedback \emph{nicht} darum, euch mitzuteilen, ob ihr ein guter oder schlechter Mensch, ein guter Redner, eine schlechte Rhetorikerin oder so seid. Solche Aussagen haben für euch keinen Lerneffekt. Stattdessen schrecken sie euch ab, Neues auszuprobieren und dabei auch einmal so genannte Fehler zu machen.

Insbesondere ist Feedback keine Grundsatzdiskussion, ob das eine oder andere Verhalten generell gut oder schlecht ist. Solche Diskussionen führt ihr besser am Abend bei einem Bierchen.

\subsection{Feedback geben}
\begin{itemize}
  \item  "`ich"' statt "`man"' oder "`wir"'
  \item die \emph{eigene} Meinung sagen
  \item GesprächspartnerIn direkt ansprechen: "`du/Sie"' statt "`er/sie"'
  \item spezifisch und konkret sein
  \item nicht verallgemeinern
  \item auf Handlungen beziehen, nicht auf Eigenschaften
  \item nicht analysieren oder psychologisieren (nicht: "`du machst das nur, weil \ldots"')
  \item Feedback möglichst unmittelbar danach geben
\end{itemize}

\subsection{Feedback bekommen}
\begin{itemize}
  \item vorher den Rahmen für das Feedback abstecken: Inhalt, Vortragstechnik, Schriftbild \ldots
  \item  gut zuhören
  \item sich nicht rechtfertigen, verteidigen oder entschuldigen
  \item Missverständnisse klären, Hintergründe erläutern
\end{itemize}
