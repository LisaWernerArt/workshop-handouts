\chapter{Lizenz}

\section*{Unter welchen Bedingungen könnt ihr dieses Handout benutzen?}
Dieses Handout ist unter einer \emph{Creative-Commons}-Lizenz lizensiert. Dies ist die \emph{Namensnennung-Weitergabe unter gleichen Bedingungen~3.0 Deutschland}\footnote{Die ausführliche Version dieser Lizenz findet ihr unter \url{http://creativecommons.org/licenses/by-sa/3.0/de/}.}. Das bedeutet, dass ihr den Reader unter diesen Bedingungen für euch kostenlos verbreiten, bearbeiten und nutzen könnt (auch kommerziell):
\begin{description}
  \item[Namensnennung.] Ihr müsst den Namen des Autors (Oliver Klee) nennen. Wenn ihr außerdem auch noch die Quelle\footnote{\url{https://github.com/oliverklee/workshop-handouts}} nennt, wäre das nett. Und wenn ihr mir zusätzlich eine Freude machen möchtet, sagt mir per E-Mail Bescheid.
  \item[Weitergabe unter gleichen Bedingungen.] Wenn ihr diesen Inhalt bearbeitet oder in anderer Weise umgestaltet, verändert oder als Grundlage für einen anderen Inhalt verwendet, dann dürft ihr den neu entstandenen Inhalt nur unter Verwendung identischer Lizenzbedingungen weitergeben.
  \item[Lizenz nennen.] Wenn ihr den Reader weiter verbreitet, müsst ihr dabei auch die Lizenzbedingungen nennen oder beifügen.
\end{description}
